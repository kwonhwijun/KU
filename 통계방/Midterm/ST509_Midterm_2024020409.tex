% Options for packages loaded elsewhere
\PassOptionsToPackage{unicode}{hyperref}
\PassOptionsToPackage{hyphens}{url}
%
\documentclass[
]{article}
\usepackage{amsmath,amssymb}
\usepackage{iftex}
\ifPDFTeX
  \usepackage[T1]{fontenc}
  \usepackage[utf8]{inputenc}
  \usepackage{textcomp} % provide euro and other symbols
\else % if luatex or xetex
  \usepackage{unicode-math} % this also loads fontspec
  \defaultfontfeatures{Scale=MatchLowercase}
  \defaultfontfeatures[\rmfamily]{Ligatures=TeX,Scale=1}
\fi
\usepackage{lmodern}
\ifPDFTeX\else
  % xetex/luatex font selection
\fi
% Use upquote if available, for straight quotes in verbatim environments
\IfFileExists{upquote.sty}{\usepackage{upquote}}{}
\IfFileExists{microtype.sty}{% use microtype if available
  \usepackage[]{microtype}
  \UseMicrotypeSet[protrusion]{basicmath} % disable protrusion for tt fonts
}{}
\makeatletter
\@ifundefined{KOMAClassName}{% if non-KOMA class
  \IfFileExists{parskip.sty}{%
    \usepackage{parskip}
  }{% else
    \setlength{\parindent}{0pt}
    \setlength{\parskip}{6pt plus 2pt minus 1pt}}
}{% if KOMA class
  \KOMAoptions{parskip=half}}
\makeatother
\usepackage{xcolor}
\usepackage[margin=1in]{geometry}
\usepackage{color}
\usepackage{fancyvrb}
\newcommand{\VerbBar}{|}
\newcommand{\VERB}{\Verb[commandchars=\\\{\}]}
\DefineVerbatimEnvironment{Highlighting}{Verbatim}{commandchars=\\\{\}}
% Add ',fontsize=\small' for more characters per line
\usepackage{framed}
\definecolor{shadecolor}{RGB}{248,248,248}
\newenvironment{Shaded}{\begin{snugshade}}{\end{snugshade}}
\newcommand{\AlertTok}[1]{\textcolor[rgb]{0.94,0.16,0.16}{#1}}
\newcommand{\AnnotationTok}[1]{\textcolor[rgb]{0.56,0.35,0.01}{\textbf{\textit{#1}}}}
\newcommand{\AttributeTok}[1]{\textcolor[rgb]{0.13,0.29,0.53}{#1}}
\newcommand{\BaseNTok}[1]{\textcolor[rgb]{0.00,0.00,0.81}{#1}}
\newcommand{\BuiltInTok}[1]{#1}
\newcommand{\CharTok}[1]{\textcolor[rgb]{0.31,0.60,0.02}{#1}}
\newcommand{\CommentTok}[1]{\textcolor[rgb]{0.56,0.35,0.01}{\textit{#1}}}
\newcommand{\CommentVarTok}[1]{\textcolor[rgb]{0.56,0.35,0.01}{\textbf{\textit{#1}}}}
\newcommand{\ConstantTok}[1]{\textcolor[rgb]{0.56,0.35,0.01}{#1}}
\newcommand{\ControlFlowTok}[1]{\textcolor[rgb]{0.13,0.29,0.53}{\textbf{#1}}}
\newcommand{\DataTypeTok}[1]{\textcolor[rgb]{0.13,0.29,0.53}{#1}}
\newcommand{\DecValTok}[1]{\textcolor[rgb]{0.00,0.00,0.81}{#1}}
\newcommand{\DocumentationTok}[1]{\textcolor[rgb]{0.56,0.35,0.01}{\textbf{\textit{#1}}}}
\newcommand{\ErrorTok}[1]{\textcolor[rgb]{0.64,0.00,0.00}{\textbf{#1}}}
\newcommand{\ExtensionTok}[1]{#1}
\newcommand{\FloatTok}[1]{\textcolor[rgb]{0.00,0.00,0.81}{#1}}
\newcommand{\FunctionTok}[1]{\textcolor[rgb]{0.13,0.29,0.53}{\textbf{#1}}}
\newcommand{\ImportTok}[1]{#1}
\newcommand{\InformationTok}[1]{\textcolor[rgb]{0.56,0.35,0.01}{\textbf{\textit{#1}}}}
\newcommand{\KeywordTok}[1]{\textcolor[rgb]{0.13,0.29,0.53}{\textbf{#1}}}
\newcommand{\NormalTok}[1]{#1}
\newcommand{\OperatorTok}[1]{\textcolor[rgb]{0.81,0.36,0.00}{\textbf{#1}}}
\newcommand{\OtherTok}[1]{\textcolor[rgb]{0.56,0.35,0.01}{#1}}
\newcommand{\PreprocessorTok}[1]{\textcolor[rgb]{0.56,0.35,0.01}{\textit{#1}}}
\newcommand{\RegionMarkerTok}[1]{#1}
\newcommand{\SpecialCharTok}[1]{\textcolor[rgb]{0.81,0.36,0.00}{\textbf{#1}}}
\newcommand{\SpecialStringTok}[1]{\textcolor[rgb]{0.31,0.60,0.02}{#1}}
\newcommand{\StringTok}[1]{\textcolor[rgb]{0.31,0.60,0.02}{#1}}
\newcommand{\VariableTok}[1]{\textcolor[rgb]{0.00,0.00,0.00}{#1}}
\newcommand{\VerbatimStringTok}[1]{\textcolor[rgb]{0.31,0.60,0.02}{#1}}
\newcommand{\WarningTok}[1]{\textcolor[rgb]{0.56,0.35,0.01}{\textbf{\textit{#1}}}}
\usepackage{graphicx}
\makeatletter
\def\maxwidth{\ifdim\Gin@nat@width>\linewidth\linewidth\else\Gin@nat@width\fi}
\def\maxheight{\ifdim\Gin@nat@height>\textheight\textheight\else\Gin@nat@height\fi}
\makeatother
% Scale images if necessary, so that they will not overflow the page
% margins by default, and it is still possible to overwrite the defaults
% using explicit options in \includegraphics[width, height, ...]{}
\setkeys{Gin}{width=\maxwidth,height=\maxheight,keepaspectratio}
% Set default figure placement to htbp
\makeatletter
\def\fps@figure{htbp}
\makeatother
\setlength{\emergencystretch}{3em} % prevent overfull lines
\providecommand{\tightlist}{%
  \setlength{\itemsep}{0pt}\setlength{\parskip}{0pt}}
\setcounter{secnumdepth}{-\maxdimen} % remove section numbering
\ifLuaTeX
  \usepackage{selnolig}  % disable illegal ligatures
\fi
\IfFileExists{bookmark.sty}{\usepackage{bookmark}}{\usepackage{hyperref}}
\IfFileExists{xurl.sty}{\usepackage{xurl}}{} % add URL line breaks if available
\urlstyle{same}
\hypersetup{
  pdftitle={Midterm ST509},
  pdfauthor={Hwijun Kwon},
  hidelinks,
  pdfcreator={LaTeX via pandoc}}

\title{Midterm ST509}
\author{Hwijun Kwon}
\date{2024-04-11}

\begin{document}
\maketitle

\hypertarget{introduction}{%
\section{1. Introduction}\label{introduction}}

\#y\_i \textbar x\_i \sim{} Poisson(\mu\emph{\{\beta\}(x\_i))
\textbackslash{} \#log(\mu}\beta{(x_i)}) = \beta\emph{0 + \beta\^{}TX
\#-\frac{1}{N} \sum\^{}\{N\}}\{i=1\}\{\{y\_\{i\}(\beta\_0 +
\beta\^{}Tx\_i) - e\^{}\{\beta\_0+\beta\^{}Tx\_i\}\}\} +
\lambda \textbar\textbar{}\beta\textbar\textbar\_1

When the response variable Y is nonnegative and represents a count, its
mean will be positive and the Poisson likelihood is often used for
inference

The l\_\{1\}-penalized negative log-likelihood is given by

Typically, We do not penalize the intercept \beta\_0. It is easy to see
that this enforces the constraint that the average fitted value is equal
to the mean response : \#\[\bar{y} = \frac{1}{N}\sum(\hat{\mu_i})\]

\hypertarget{competing-methods}{%
\section{2. Competing Methods}\label{competing-methods}}

\hypertarget{unpenalized-poisson-regression}{%
\subsection{1. Unpenalized Poisson
Regression}\label{unpenalized-poisson-regression}}

log(\mu\_\beta{(x_i)}) = \beta\emph{0 + \beta\^{}\{TX\} \textbackslash{}
\mu}\beta{(x_i)} = \lambda\_i = exp(\beta\_0 + \beta\^{}\{TX\})
\textbackslash{}

P(Y = y\_i \textbar{} X=x\_i)
=\frac{\exp^{\lambda_i} \lambda_i}{y_i!}\^{}\{y\_i\} \textbackslash{} =
\frac{\exp^{-exp(\beta_0 + \beta^TX)} *exp(\beta_0 + \beta^TX)}{y_i!}\^{}\{y\_i\}
\textbackslash{} X\_i\beta = \beta\_0 + \beta\^{}TX\_i \textbackslash{}
L(\beta) = \prod\emph{1\^{}n
\frac{e^{-exp(X_i\beta)} exp(X_i\beta)^{y_i}}{y_i!} \textbackslash{}
l(\beta) = \sum\emph{1\^{}n -exp(X\_i\beta) + \sum y\_i(X\_i\beta) -
\sum log(y\_i) \textbackslash{} \beta}\{OLS\} =
argmin}\{\beta\}l'(\beta)

Poisson Regression Solves

\hypertarget{ridge-solves}{%
\subsubsection{2. Ridge Solves}\label{ridge-solves}}

\#\hat{\beta}\_\{ridge\} =
argmin\frac{1}{N}\sum(y\_i-\beta\_0-\beta\textsuperscript{Tx\_i)}2 +
\frac{\lambda}{2} \textbar\textbar{}\beta\textbar\textbar\^{}2\_2

\hypertarget{lasso-solves}{%
\subsubsection{3. Lasso Solves}\label{lasso-solves}}

\#\hat{\beta}\_\{ridge\} =
argmin\frac{1}{N}\sum(y\_i-\beta\_0-\beta\textsuperscript{Tx\_i)}2 +
\frac{\lambda}{2} \textbar\textbar{}\beta\textbar\textbar\_1

\hypertarget{elastic-net}{%
\subsubsection{3. Elastic Net}\label{elastic-net}}

\hypertarget{section}{%
\subsubsection{4.}\label{section}}

\hypertarget{simulation-set-up}{%
\section{3. Simulation Set up}\label{simulation-set-up}}

\hypertarget{generating-data}{%
\subsection{1. Generating Data}\label{generating-data}}

\begin{Shaded}
\begin{Highlighting}[]
\FunctionTok{library}\NormalTok{(MASS)}
\NormalTok{generate\_data }\OtherTok{\textless{}{-}} \ControlFlowTok{function}\NormalTok{(n, beta, beta0, p, nu) \{}
\NormalTok{  mu }\OtherTok{\textless{}{-}} \FunctionTok{rep}\NormalTok{(}\DecValTok{0}\NormalTok{, p)}
\NormalTok{  sigma }\OtherTok{\textless{}{-}} \FunctionTok{outer}\NormalTok{(}\DecValTok{1}\SpecialCharTok{:}\NormalTok{p, }\DecValTok{1}\SpecialCharTok{:}\NormalTok{p, }\AttributeTok{FUN =} \ControlFlowTok{function}\NormalTok{(i, j) nu}\SpecialCharTok{\^{}}\NormalTok{(}\FunctionTok{abs}\NormalTok{(i}\SpecialCharTok{{-}}\NormalTok{j)))}
\NormalTok{  x }\OtherTok{\textless{}{-}} \FunctionTok{mvrnorm}\NormalTok{(}\AttributeTok{n=}\NormalTok{n, }\AttributeTok{mu=}\NormalTok{mu, }\AttributeTok{Sigma=}\NormalTok{sigma)}
\NormalTok{  mu\_x }\OtherTok{\textless{}{-}} \FunctionTok{exp}\NormalTok{(beta0 }\SpecialCharTok{+}\NormalTok{ x }\SpecialCharTok{\%*\%}\NormalTok{ beta)}
\NormalTok{  y }\OtherTok{\textless{}{-}} \FunctionTok{rpois}\NormalTok{(n, mu\_x)}
  \FunctionTok{return}\NormalTok{(}\FunctionTok{list}\NormalTok{(}\AttributeTok{x =}\NormalTok{ x, }\AttributeTok{y =}\NormalTok{ y))}
\NormalTok{\}}
\FunctionTok{set.seed}\NormalTok{(}\DecValTok{2024020409}\NormalTok{)}
\NormalTok{train\_data }\OtherTok{\textless{}{-}} \FunctionTok{generate\_data}\NormalTok{(}\DecValTok{500}\NormalTok{, }\FunctionTok{c}\NormalTok{(}\DecValTok{1}\NormalTok{, }\DecValTok{1}\NormalTok{, }\DecValTok{1}\NormalTok{), }\DecValTok{1}\NormalTok{, }\DecValTok{3}\NormalTok{, }\FloatTok{0.7}\NormalTok{)}
\NormalTok{test\_data }\OtherTok{\textless{}{-}} \FunctionTok{generate\_data}\NormalTok{(}\DecValTok{500}\NormalTok{, }\FunctionTok{c}\NormalTok{(}\DecValTok{1}\NormalTok{, }\DecValTok{1}\NormalTok{, }\DecValTok{1}\NormalTok{), }\DecValTok{1}\NormalTok{, }\DecValTok{3}\NormalTok{, }\FloatTok{0.7}\NormalTok{)}
\end{Highlighting}
\end{Shaded}

\begin{Shaded}
\begin{Highlighting}[]
\NormalTok{poisson\_fit }\OtherTok{\textless{}{-}}\NormalTok{ glmnet}\SpecialCharTok{::}\FunctionTok{glmnet}\NormalTok{(train\_data}\SpecialCharTok{$}\NormalTok{x, train\_data}\SpecialCharTok{$}\NormalTok{y, }\AttributeTok{family=}\FunctionTok{poisson}\NormalTok{(}\AttributeTok{link=}\StringTok{"log"}\NormalTok{), }\AttributeTok{alpha =} \DecValTok{1}\NormalTok{)}

\NormalTok{predictions }\OtherTok{\textless{}{-}} \FunctionTok{predict}\NormalTok{(poisson\_fit, }\AttributeTok{newx=}\NormalTok{test\_data}\SpecialCharTok{$}\NormalTok{x, }\AttributeTok{type=}\StringTok{"response"}\NormalTok{)}

\NormalTok{mse }\OtherTok{\textless{}{-}} \FunctionTok{mean}\NormalTok{((test\_data}\SpecialCharTok{$}\NormalTok{y }\SpecialCharTok{{-}}\NormalTok{ predictions)}\SpecialCharTok{\^{}}\DecValTok{2}\NormalTok{)}
\NormalTok{variance }\OtherTok{\textless{}{-}} \FunctionTok{var}\NormalTok{(predictions)}
\NormalTok{mean\_prediction }\OtherTok{\textless{}{-}} \FunctionTok{mean}\NormalTok{(predictions)}
\NormalTok{bias }\OtherTok{\textless{}{-}}\NormalTok{ mean\_prediction }\SpecialCharTok{{-}} \FunctionTok{mean}\NormalTok{(test\_data}\SpecialCharTok{$}\NormalTok{y)}

\CommentTok{\#list(MSE = mse, Variance = variance, Bias = bias)}
\CommentTok{\#predictions}
\end{Highlighting}
\end{Shaded}

\hypertarget{result}{%
\section{4. Result}\label{result}}

\hypertarget{discussion}{%
\section{5. Discussion}\label{discussion}}

\end{document}
